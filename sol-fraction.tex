\documentclass[11pt, oneside]{article}   	% sets class to "article"

\title{Sol Fraction}

\begin{document}

\maketitle 							% Places title in document.
\date{}							% Activate to display a given date or no date

\begin{enumerate}

\item Make sure specimen is clean and free of any contaminants.
\item If not completely dry, dry either under heated or non-heated condition for a set time.
\item Tare a clean weigh boat or piece of weigh paper.
\item Mass specimen in weigh boat or on weigh paper and record.
\item Dry specimen further.
\item Mass specimen again and record.
\item If mass has decreased significantly, continue to dry and repeat massing.
\item Fill solvent compatible container with solvent and carefully add specimen so as to not change its mass.
\item Seal container to prevent large evaporation of solvent.
\item Set aside container at elevated or room temperature for set time.
\item At appropriate time carefully remove sample and place into solvent compatible weigh boat or weigh paper.
\item Dry specimen under heated or non-heated condition for a set time.
\item Tare a clean weigh boat or piece of weigh paper.
\item Mass specimen in new weigh boat or weigh paper and record.
\item If mass has decreased significantly, continue to dry and repeat massing.
\item Once mass has leveled out, apply the following equation to determine sol-fraction, $f_{sol}$:

    \begin{equation}
    	f_{sol} = \frac{m_{i} - m_{f}}{m_{i}}
    \end{equation}
    
    \noindent
    where, $m_{i}$, is the initial fully dried mass, and $m_{f}$, is the post extraction fully dried mass.  
    
\item The \% sol fraction, $F_{sol}$ can then be found by:

    \begin{equation}
    	F_{sol} = f_{sol}*100\%
    \end{equation}

\end{enumerate}

\end{document}